\documentclass[11pt]{article}

\usepackage{fullpage}

\begin{document}

\title{ARM Checkpoint}
\author{Taariq Pala, Satardu Sanyal, Akanksha Sharma, Cyprien Roche}

\maketitle

\section{Group Organisation}

%A statement on how you’ve split the work between group members and how you are co-ordinating your work.
Every group member was given a sub-problem of the overall task. Part one and part two were split according to everyone's programming abilities. This meant that each member would spend a roughly equal amount of time thinking about their problem. We chose to split time evenly over quantity. Those more confidents covered more ground and allowed the group to move swiftly through the two parts, in reasonable time. This has proven to be a beneficial strategy regarding the deadline for this report. Every member was very supportive and offered help to other members in need. After looking at our respective parts, we each explained what worked and what was going wrong to other group members. All the group was kept up to date with the most recent code. Hence, everyone's thoughts was heard, which allowed the group to chose the best implementation, having come to a consensus.
\\
\\
In terms of coordination, although the majority of our discussion was done online, we have been meeting regularly in the laboratories. Our communication has been good. We could improve by adding extra comments to our code which would accelerate the explanations to other group members. Those extra comments could easily be removed towards the end of the project.
\\
Every member has their own branch on git. We could improve by creating branches for each sub-problems individually and rebase when possible, rather than each member working on one branch of their own throughout the project. This would make our git repository clearer as to what parts have been completed and which are in the process of being solved. However, git is not easily used to its full potential and requires a lot of time and mistakes to master. We have been spending some time on git already and regarding the basics of git group project, we are doing well. 


%A discussion on how well you think the group is working and how you imagine it might need to change for the later tasks.

\section{Implementation Strategies}

%How you’ve structured your emulator, and what bits you think you will be able to reuse for the assembler.

We have been following the specification as close as we could. Namely, our emulator has a structure called current state which is made up of an array of unsigned one byte integers called memory, an array of signed 32 bit integers called registers, a fetched and a decoded instruction. The last two which are structs of their own. Memory and registers are initialised to zero. Memory is (64KB) 65536 bytes in size as required by the specification. This memory fills up when a binary file containing instructions is opened and read. Each instruction is then fetched, decoded and executed according to the three stage pipelining as per the specification.  


%A discussion on implementation tasks that you think you will find difficult / challenging later on, and how you are working to mitigate these.

\end{document}
